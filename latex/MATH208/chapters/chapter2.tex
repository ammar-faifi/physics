\chapter{Linear Systems and Matrices} % (fold)
\label{cha:Linear Systems and Matrices}
    \section{Introduction to Linear Systems}
        Consider the following equation 
        \[ ax + by  +cz = d \qquad \text{or}\, ax + by = c \]
        each one is called a \textbf{linear equation}; because the \textbf{variables} are power of one.

        \bulletpar A \textbf{system} of linear equations (or \textbf{linear system}) is a finite collections of linear equations involving ceratin variables 
        (\textit{unknowns}). for examole, \[ a_1 x + b_1 y = c_1 \\ a_2 x + b_2 y = c_2\] which is a linear system in two unkowns.
        A \textbf{solution} is a pair of $(x, y)$ that satisfies \textit{both} equations \textit{simultaneously}.

        \begin{itemize}
            \item A \textbf{consistent} systems have at least one solution.
            \item A \textbf{inconsistent} systems have no solution.
        \end{itemize}
        
        \noindent
        For a linear system, we havr three possiblities:
        \begin{enumerate}
            \item exactly on solution.
            \item no solution.
            \item infinitely many solutions.
        \end{enumerate}

        \paragraph{The Method of Eliminations}
            \begin{enumerate}
                \item Multiply one equation with nonzero number.
                \item Interchange two equations.
                \item Add a constant multiple of one equation to another equation.
            \end{enumerate}

        \paragraph{A Differential Equation}
            Consider this equation 
            \begin{align}
                \label{eq:differential application:1}
                y(x) = Ae^{nx} + Be^{-nx}
            \end{align} 
            after differentiation we get $$ y'(x) = nAe^{nx} - n Be^{-nx} $$
            also
            $$ y''(x) = n^2 A e^{nx} + n^2 B e^{-nx} = n^2 y(x)$$
            Thus Eq.\eqref{eq:differential application:1} is a solution to the differential equation
            \begin{align}
                y'' - n^2y = 0 
            \end{align}
            to find the constant $A$ and $B$ we need to solve a linear systme after applying the initial condition.

    \section{Matrices and Gaussian Elimination}
        A general system of $m$ linear equation in $n$ variables $x_1, x_2, \dots, x_n$ written in form 
        \begin{align}
            \begin{aligned}
                \label{eq:general system}
                a_{11}x_1 + a_{12}x_2 + , \dots + a_{1n}x_n &= b_1 \\
                a_{21}x_1 + a_{22}x_2 + , \dots + a_{2n}x_n &= b_2 \\
                                                            &\vdots \\
                a_{m1}x_1 + a_{m2}x_2 + , \dots + a_{mn}x_n &= b_m \\
            \end{aligned}
        \end{align}
        The \textbf{coefficient matrix} of this linea system is written in $m \times n$ matrix 
        \begin{align*}
            A = \begin{bmatrix}
                a_{11} & a_{12} & \dots & a_{1n} \\
                a_{21} & a_{22} & \dots & a_{2n} \\
                \vdots & \vdots & \ddots & \vdots \\
                a_{m1} & a_{m2} & \dots & a_{mn}
            \end{bmatrix}
        \end{align*}
        For large number of entries we can denote $a_{ij}$ as $a(i, j)$. The \textbf{constant matrix} is 
        \begin{align*}
            b = \begin{bmatrix}
                b_1 \\ b_2 \\ \vdots \\ b_m
            \end{bmatrix}
        \end{align*}
        This is called a \textbf{vector} (\textbf{column}) matrix. Combining both matrices we get a $m \times (n+1)$ \textbf{augmented matrix}
        \begin{align*}
            [A\, b] = \begin{bmatrix}
                a_{11} & a_{12} & \dots & a_{1n} & b_1 \\
                a_{21} & a_{22} & \dots & a_{2n} & b_2 \\
                \vdots & \vdots & \ddots & \vdots & \vdots \\
                a_{m1} & a_{m2} & \dots & a_{mn} & b_m
            \end{bmatrix}
        \end{align*}

        \paragraph{Elemnetary Row Operations} % (fold)
        \label{par:Elemnetary Row Operations}
        As we did before for solving a linear sustem, we can generalize that for augmented matrices as 
        \coloredeq{elemetary row operations}{\parbox[t]{0.8\linewidth}{
            \textbf{Elemnetary Row Operations}
            \begin{enumerate}
                \item Multiply any row by a nonzero number.
                \item Interchange two rows.
                \item Add a constant multiple of one row to another row.
            \end{enumerate}
        }}
        
        \coloredeq{Row-Equivalent Matrices}{\parbox[t]{0.7\linewidth}{
            \textbf{\color{g2}Row-Equivalent Matrices}\\
            Two matrices are called \textbf{tow eauivalent} if one can be obtained from the pther by a finite sequence of elementary row opearations
        }}

        \coloredeq{theorem1}{\parbox[t]{0.7\linewidth}{
            \textbf{\color{g2}THEOREM1:  Equivalent Systems and Equivalent Matrices}\\ \newline
            If the augmneted coefficient matrices of two systems are row equivalent, then the two systems have the same solution set.
        }}
        % paragraph Elemnetary Row Operations (end)

        \paragraph{Echelon Matrices} % (fold)
        \label{par:Echelon Matrices}
        Let us see how sould the augmented matrix look after some elementry row operations,
        
        \coloredeq{theorem1}{\parbox[t]{0.8\linewidth}{
            \textbf{\color{g2}DEFINITION:  Echelon Matrix}\\ \newline
            The matrix $E$ is called an \textbf{echelon matrix} provided it has the following two properties:
            \begin{enumerate}
                \item Every row that consists of entirely zeros (if any) liew \textit{beneath} every row that contains a nonzero element.
                \item In each  row contains nonzero element, the \textit{first} nonzero element lies strictly to the right of the first nonzero element in the preceding 
                row (if there is a preseding row).
            \end{enumerate}
        }}
        % paragraph Echelon Matrices (end)






% chapter Linear Systems and Matrices (end)